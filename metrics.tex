\documentclass[titlepage, hidelinks, 12pt]{article}


\usepackage{lipsum}
\usepackage{hyperref}
\usepackage{palatino}

\usepackage{chngcntr}
\counterwithin{figure}{section}

\usepackage{setspace}
%\usepackage{indentfirst} %tex default is no indent on first paragraph after heading
\usepackage{url}
\usepackage{amsmath, amssymb, amsfonts, amsthm}
\usepackage{float}
\usepackage{subfig}
\usepackage{graphicx}
\usepackage{environ, enumerate}
%\usepackage{mathbbol}
%\DeclareSymbolFontAlphabet{\amsmathbb}{AMSb}
\graphicspath{ {images/} }
\providecommand{\keywords}[1]{\textbf{\textit{Keywords---}} #1} 
\usepackage[format=plain,
            labelfont={bf, it},
            textfont=it]{caption}

%%%%%%%%%
% indentation
%%%%%%%%%

\setlength\parindent{0pt}
\setlength{\parskip}{\baselineskip}

\setlength{\voffset}{-1cm}
\setlength{\textwidth}{17cm}
\addtolength{\textheight}{2cm}
\setlength{\footskip}{1cm}
\addtolength{\oddsidemargin}{-2cm}
\addtolength{\evensidemargin}{-2cm}

\widowpenalty10000
\clubpenalty10000

%Def, Lemma, Theorem, Corollary environment
\theoremstyle{plain}
\newtheorem{theorem}{Theorem}[subsection]
\newtheorem{corollary}[theorem]{Corollary}
\newtheorem{lemma}[theorem]{Lemma}
\newtheorem{proposition}[theorem]{Proposition}
%\newtheorem*{proof}{Proof}
\theoremstyle{remark}
\newtheorem*{remark}{Remark}
\newtheorem*{example}{Example}
\theoremstyle{definition}
\newtheorem{definition}[theorem]{Definition}

%New commands
\newcommand{\Q}{\mathbb{Q}}
\newcommand{\Z}{\mathbb{Z}}
\newcommand{\N}{\mathbb{N}}
\newcommand{\R}{\mathbb{R}}

%New math operators
\DeclareMathOperator{\diag}{diag}

\begin{document}


\title{Some metrics on the space of cancer treatments}
\author{PIMS $\text{Math}^{\text{Industry}}$ Team 3 (I think)}
\maketitle






\newpage

\section{Preamble and some notation}
Let the universal drug set be $\mathcal{D} = \left\{ d_1, d_2, \ldots \right\}$. Some of these drugs may be prescribed to treat a cancer: 
a specific cancer treament is some  $T = \left\{ d_{i_1}, \ldots, d_{i_n} \right\}\subseteq \mathcal{D}$. 
The set of all treatments is $\mathcal{T} = \mathcal{P}(\mathcal{D})$, the power set of $\mathcal{D}$. 

In an ideal test setting, an experimental treatment $T$ is given to a large sample and we check: did the treatment improve outcomes over
standard treatments? In practice, however, experimental treatments may be given imperfectly. That is, a treatment $T' \approx T$ may be given,
where $\approx$ is a vague notion of similar treatment.

One purpose of this project is to define a metric $d$ on $\mathcal{T}$ quantifying similarity of treatment. Or perhaps it isn't a metric that
we want\dots it probably makes statistical analysis easier if $d(T, T) = 1$ for any $T$, and if $d(T, T')$ gets smaller as $T'$ differs more
from $T$. So the Jaccard index $J(A, B) = |A\cap B|/|A\cup B|$ would be a reasonable first guess. 

\section{Matching score}
A confounding factor is that many drugs can target the same aberration. That is, while aionics might recommend drug $d$, a physician may
prescribe $d'$ without issue.  The Jaccard index is then probably not the greatest scoring available. Instead, we'll want to collapse drugs
down to the thing (this is vague, but I don't know the correct term) they target. 

The Matching Score basically does this. Vaguely, the matching score is defined by:

\begin{equation}
    \text{Matching Score }:= \frac{\text{\# of aberrations addressed by prescribed drug regimen}}{\text{\# of aberrations in cancerous tissue}}
    \label{eqn:matchingScore}
\end{equation}

Of course, the exact number of aberrations present and addressed is sometimes ambiguous, and certain drugs work better together than others.
The paper ``Molecular profiling of cancer patients enables personalized combination therapy: the I-PREDICT study'' runs through some of the
considerations when assigning a matching score. Bo quotes a similar (though not identical) description on Slack. 













\section{Considerations when constructing a score}
Drug choice seems to be one layer of abstraction too far. Rather, when evaluating how closely a drug regimen comes to desired, what we
wish to understand is: how closely does the regimen attack the issues present in the cancerous tissue? Further, how intensely shoudld we attack?
If I remember correctly, aionics provides the following. To each cancer profile $C$ is associated a symptom/treatment table: 

\begin{table}[h]
    \centering
\begin{tabular}{l|l|l|l}
aberration           & intensity       & importance   & drug options            \\\hline
name\_of\_aberration & $x\in\R_{\ge0}$ & $w\in(0, 1)$ & $d_1, d_2, \ldots, d_k$ \\
\end{tabular}
\end{table}

We can construct a similar table (first two columns, at least) for each drug regimen $T$. 
The question, now, is what is a reasonable scoring function $\sigma(C, T)$? Here $\sigma(C, T)$ is a scoring of closeness, evaluating
how well treatment $T$ attacks cancer profile $C$. 

As a first step, consider how well a regimen attacks a specific aberration. If we want to attack an aberration with
intensity $x$, and drug combination $T$ attacks with intensity $x_T$, then we'll let $\nu(x, x_T)$ be the intensity adjustment (adjusting
treatment effectiveness for how close to prescribed intensity we attack). For instance, we could let
\begin{equation}
    \nu(x, x_T) := e^{-(x-x_T)^2}.
    \label{eqn:intensityAdjustment}
\end{equation}
From here, it makes sense to define the score:

\begin{equation}
    \sigma(C, T) = \sum\limits_{a\in\text{Aberrations}} w_a\cdot \nu(x_a, x_{a,T})
    \label{eqn:ourMatchingScore}
\end{equation}






\end{document}



